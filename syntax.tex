
No te olvides de echarle un ojo a la página con los cinco errores de escritura más frecuentes\footnote{\url{http://www.tallerdeescritores.com/errores-de-escritura-frecuentes}}.

Aconsejo a todo el mundo que mire y se inspire en memorias pasadas.
Las memorias de los proyectos que he llevado yo están (casi) todas almacenadas en mi web del GSyC\footnote{\url{https://gsyc.urjc.es/~grex/pfcs/}}.

En mayo de 2023 me apunté a un curso de innovación docente donde nos pidieron hacer un podcast con temática docente. Aproveché entonces para hacer un podcast de unos 30 minutos donde en los primeros quince minutos introducía LaTeX y la memoria, y en los segundos hacía hincapién en aquellas cosas que más os cuestan utilizar en la memoria: las figuras, las tablas y las citas. Podéis escuchar el podcast en Internet\footnote{\url{https://podcasters.spotify.com/pod/show/gregorio-robles9/episodes/Tu-memoria-de-Trabajo-Fin-de-Grado-o-de-Mster-en-LaTeX-e23hucr/a-a58kp2}}.


\section{Sección}
\label{sec:seccion}

Esto es una sección, que es una estructura menor que un capítulo. 

Por cierto, a veces me comentáis que no os compila por las tildes.
Eso es un problema de codificación.
Al guardar el archivo, guardad la codificación de ``ISO-Latin-1'' a ``UTF-8'' (o viceversa) y funcionará.

\subsection{Estilo}
\label{subsec:estilo}

Recomiendo leer los consejos prácticos sobre escribir documentos científicos en \LaTeX \ de Diomidis Spinellis\footnote{\url{https://github.com/dspinellis/latex-advice}}.

Lee sobre el uso de las comas\footnote{\url{http://narrativabreve.com/2015/02/opiniones-de-un-corrector-de-estilo-11-recetas-para-escribir-correctamente-la-coma.html}}. 
Las comas en español no se ponen al tuntún.
Y nunca, nunca entre el sujeto y el predicado (p.ej. en ``Yo, hago el TFG'' sobre la coma).
La coma no debe separar el sujeto del predicado en una oración, pues se cortaría la secuencia natural del discurso.
No se considera apropiado el uso de la llamada coma respiratoria o \emph{coma criminal}.
Solamente se suele escribir una coma para marcar el lugar que queda cuando omitimos el verbo de una oración, pero es un caso que se da de manera muy infrecuente al escribir un texto científico (p.ej. ``El Real Madrid, campeón de Europa'').

A continuación, viene una figura, la Figura~\ref{figura:foro_hilos}. 
Observarás que el texto dentro de la referencia es el identificador de la figura (que se corresponden con el ``label'' dentro de la misma). 
También habrás tomado nota de cómo se ponen las ``comillas dobles'' para que se muestren correctamente. 
Nota que hay unas comillas de inicio (``) y otras de cierre (''), y que son diferentes.
Volviendo a las referencias, nota que al compilar, la primera vez se crea un diccionario con las referencias, y en la segunda compilación se ``rellenan'' estas referencias. 
Por eso hay que compilar dos veces tu memoria.
Si no, no se crearán las referencias.



 \begin{figure}
    \centering
    \includegraphics[bb=0 0 800 600, width=12cm, keepaspectratio]{img/foro1}
    \caption{Página con enlaces a hilos}
    \label{figura:foro_hilos}
 \end{figure}


A continuación un bloque ``verbatim'', que se utiliza para mostrar texto tal cual.
Se puede utilizar para ofrecer el contenido de correos electrónicos, código, entre otras cosas.


{\footnotesize
\begin{verbatim}
    From gaurav at gold-solutions.co.uk  Fri Jan 14 14:51:11 2005
    From: gaurav at gold-solutions.co.uk (gaurav_gold)
    Date: Fri Jan 14 19:25:51 2005
    Subject: [Mailman-Users] mailman issues
    Message-ID: <003c01c4fa40$1d99b4c0$94592252@gaurav7klgnyif>

    Dear Sir/Madam,
    How can people reply to the mailing list?  How do i turn off
    this feature? How can i also enable a feature where if someone
    replies the newsletter the email gets deleted?
    Thanks

    From msapiro at value.net  Fri Jan 14 19:48:51 2005
    From: msapiro at value.net (Mark Sapiro)
    Date: Fri Jan 14 19:49:04 2005
    Subject: [Mailman-Users] mailman issues
    In-Reply-To: <003c01c4fa40$1d99b4c0$94592252@gaurav7klgnyif>
    Message-ID: <PC173020050114104851057801b04d55@msapiro>

    gaurav_gold wrote:
    >How can people reply to the mailing list?  How do i turn off
    this feature? How can i also enable a feature where if someone
    replies the newsletter the email gets deleted?

    See the FAQ
    >Mailman FAQ: http://www.python.org/cgi-bin/faqw-mm.py
    article 3.11
\end{verbatim}
}



\section{Estructura de la memoria}
\label{sec:estructura}


\begin{figure}
  \centering
  \includegraphics[width=9cm, keepaspectratio]{img/arquitectura.png}
  \caption{Estructura del parser básico}
  \label{fig:arquitectura}
\end{figure}




En esta sección se debería introducir la estructura de la memoria. 

Así:


\begin{itemize}
  \item En el primer capítulo se hace una intro al proyecto.
  
  \item En el capítulo~\ref{chap:objetivos} (ojo, otra referencia automática) se muestran los objetivos del proyecto.
  
  \item A continuación se presenta el estado del arte en el capítulo~\ref{chap:estado}.
  
  \item \ldots
\end{itemize}